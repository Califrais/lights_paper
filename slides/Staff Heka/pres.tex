\documentclass{beamer}

\mode<presentation>
{
  \usetheme{Goettingen}      
  \usecolortheme{default}
  \usefonttheme{default}
  \setbeamertemplate{navigation symbols}{}
  \setbeamertemplate{caption}[numbered]
} 

\usepackage[english]{babel}
\usepackage[utf8x]{inputenc}
\usepackage{amsmath}
\usepackage{dsfont}
\usepackage{booktabs,multirow,array}
\usepackage{algorithm}
\usepackage{caption}
\usepackage{subcaption}
\usepackage{graphicx}
\usepackage{url}
\usepackage{xcolor}
\usepackage{moresize}
\usepackage{lmodern}
\usepackage{mathrsfs}

\usepackage[numbers]{natbib}
\renewcommand\bibsection{\section[]{\refname}}

\definecolor{blue_pres}{RGB}{51,50,178}
\definecolor{darkpastelgreen}{rgb}{0.01, 0.75, 0.24}

\hypersetup{
    colorlinks,
    citecolor=orange,
    linkcolor=blue_pres,
    backref=true,
    urlcolor=darkpastelgreen
}

\makeatletter
\let\@@magyar@captionfix\relax
\makeatother

\DeclareCaptionFormat{myformat}{#3}
\captionsetup[algorithm]{format=myformat}
\usepackage{algpseudocode}

%%%%%%%%%%%%%%%%%%%%
%%% New commands %%%
%%%%%%%%%%%%%%%%%%%%

\DeclareMathOperator{\TV}{TV}
\DeclareMathOperator{\bina}{bina}
\DeclareMathOperator{\argmin}{argmin}

\newcommand{\dd}{\mathrm{d}}
\newcommand{\bC}{\mathbb C}
\newcommand{\bD}{\mathbb D}
\newcommand{\bP}{\mathbb P}
\newcommand{\bX}{\boldsymbol X}
\newcommand{\XB}{{\boldsymbol X}^B}
\newcommand{\E}{\mathbb E}
\newcommand{\N}{\mathbb N}
\newcommand{\bigO}{\mathcal{O}}
\newcommand{\C}{\mathbb{C}}
\newcommand{\D}{\mathbb{D}}
\newcommand{\R}{\mathbb R}
\newcommand{\cA}{\mathcal A}
\newcommand{\cB}{\mathcal B}
\newcommand{\cH}{\mathcal H}
\newcommand{\cS}{\mathcal S}
\newcommand{\cY}{\mathcal Y}
\newcommand{\cC}{\mathcal C}
\newcommand{\cG}{\mathcal G}
\newcommand{\cN}{\mathcal N}
\newcommand{\cU}{\mathcal U}
\renewcommand{\P}{\mathds{P}}
\newcommand{\Gb}{\bar G}
\newcommand{\Fb}{\bar F}
\newcommand{\Ub}{\bar U}
\newcommand{\by}{\boldsymbol y}
\newcommand{\bz}{\boldsymbol z}
\newcommand{\bdelta}{\boldsymbol \delta}
\newcommand{\bSigma}{\textbf{$\Sigma$}}
\newcommand{\norm}[1]{\|#1\|}
\newcommand{\pluseq}{\mathrel{+}=}
\newcommand{\minuseq}{\mathrel{-}=}
\newcommand{\bcdot}{\raisebox{-0.25ex}{\scalebox{1.2}{$\cdot$}}}
\newcommand{\ind}[1]{\mathds{1}_{#1}}
\newcommand\independent{\protect\mathpalette{\protect\independenT}{\perp}} 
\newcommand{\cD}{\mathcal D}
\newcommand{\cQ}{\mathcal Q}
\newcommand{\cR}{\mathcal R}
\renewcommand{\P}{\mathds P}


\definecolor{blue_pres}{RGB}{51,50,178}

\newcommand\Algphase[1]{%
\vspace*{-.7\baselineskip}\Statex\hspace*{\dimexpr-\algorithmicindent-2pt\relax}\rule{\textwidth}{0.4pt}%
\vspace*{-.4\baselineskip}\Statex\hspace*{\algorithmicindent}\textbf{#1}%
\vspace*{-.7\baselineskip}\Statex\hspace*{\dimexpr-\algorithmicindent-2pt\relax}\rule{\textwidth}{0.4pt}%
}
\newcommand*\rot{\rotatebox{90}}
\newcommand*\OK{\ding{51}}

\def\independenT#1#2{\mathrel{\rlap{$#1#2$}\mkern2mu{#1#2}}}
\newtheorem{hyp}{Hypothesis}


\title[Lights \\ \insertframenumber/\inserttotalframenumber \vspace{-1cm}]{}

\institute[shortinst]{
  \vspace{-2.2cm} \\
  \includegraphics[width=2cm]{figures/SU} \hspace{.2cm}
  \includegraphics[width=1cm]{figures/cnrs} \hspace{.1cm}
  \includegraphics[width=2.3cm]{figures/UniversiteParis.png}
  \includegraphics[width=1.7cm]{figures/lpsm_logo.jpeg}
  \includegraphics[width=1.4cm]{figures/Califrais_logo} \\
  \vspace{.6cm}
  \textcolor{blue_pres}{\Large HeKA Staff Meeting}\\
  \vspace{.5cm}
  \large
  \textbf{Simon Bussy}\\
  \vspace{.25cm}
  \scriptsize
  \href{mailto:simon.bussy@gmail.com}{\texttt{simon.bussy@gmail.com}}\\
  \vspace{.4cm}
  \small
  \textbf{Lights: a generalized joint model for high-dimensional multivariate longitudinal data and censored durations} \\
  \vspace{.3cm}
  \tiny
  S. Bussy $^{(1,2)}$, V.T. Nguyen$^{(1,2)}$, A. Barbieri $^{(3)}$, S. Zohar $^{(4)}$, A.S. Jannot $^{(4,5)}$\\

  \vspace{.4cm}  

  \tiny
  
  $^{(1)}$LOPF, Califrais’ Machine Learning Lab, Paris, France \\
  $^{(2)}$LPSM, UMR 8001, CNRS, Sorbonne University, Paris, France \\
  $^{(3)}$INSERM, UMR 1219, Bordeaux Population Health Research Center, Univ. Bordeaux, France \\
  $^{(4)}$INSERM, UMRS 1138, Centre de Recherche des Cordeliers, Paris, France \\
  $^{(5)}$Biomedical Informatics and Public Health Department, EGPH, APHP, Paris, France
}

\date{}

\begin{document}

\begin{frame}
  \titlepage
\end{frame}

\section{Introduction}

\begin{frame}[noframenumbering]
\Large \centering
\textcolor{blue_pres}{I.} Introduction
\end{frame}

\subsection{Overview}

\begin{frame}{Overview}

\begin{itemize}
  \item<1-> Deal with the problem of joint modeling of longitudinal data and censored durations
  \item<2-> Large number of both longitudinal and time-independent features are available
  \item<3-> Flexibility in modeling the dependency between the longitudinal features and the event time with appropriate penalties
  \item<4-> Inference achieved using an efficient and novel Proximal Quasi-Newton Expectation Maximization algorithm
\end{itemize}

\end{frame}

\subsection{Use cases}

\begin{frame}{Use cases}

\small
\begin{itemize}
  \item<1-> Predict the risk for an event of interest to occur quickly, taking into account simultaneously a huge number of longitudinal signals in a high-dimensional context
  \item<2-> Provides powerful interpretability by automatically pinpointing significant time-dependent and time-independent features
\end{itemize}

\begin{block}<3->{Real-time decision support}
\begin{itemize}
  \item<3-> Medical context $\rightarrow$ event of interest: survival time, re-hospitalization, relapse or disease progression ; longitudinal data: biomarkers or vital parameters measurements
  \item<4-> Customer's satisfaction monitoring context $\rightarrow$ event of interest: time when a client churns ; longitudinal data: the client's activity recorded from account opening throughout the duration of the business relationship
\end{itemize}
\end{block}

\end{frame}

\subsection{Framework}

\begin{frame}{High-dimensional framework}

\small
\begin{itemize}
  \item<1-> Survival analysis
  \begin{center}
  $T = T^\star \wedge C \quad \text{and} \quad \Delta = \ind{\{T^\star \leq C\}}$
  \end{center}
  \item<2-> Time-independent features $X \in \R^p$ with $p \gg n$
  \item<3-> $L$ longitudinal outcomes such that $L \gg n$ and \[Y(t) = \big(Y^1(t), \ldots, Y^L(t) \big)^\top \in \R^L\]
  \item<4-> Heterogeneity of the population: latent subgroups \[G \in \{ 0, \ldots, K-1\}\]
  \item<5-> Softmax link function for the latent class membership probability given time-independent features
  \begin{center}
  $\pi_{\xi_k}(x) = \P[G=k|X=x] = \dfrac{e^{x^\top\xi_k}}{\sum_{k=0}^{K-1}e^{x^\top\xi_k}}$
  \end{center}
\end{itemize}

\end{frame}

\section{Model}

\begin{frame}[noframenumbering]
\Large \centering
\textcolor{blue_pres}{II.} Model
\end{frame}

\subsection{Submodels}

\begin{frame}{Submodels}

\small
\begin{block}<1->{Group-specific marker trajectories}
\begin{itemize}
  \footnotesize
  \item<1-> $h_l\big(\E[Y^l(t)|b^l, G=k]\big) = m_k^l(t) = u^l(t)^\top\beta_k^l + v^l(t)^\top b^l$\\
  with fixed effect parameters $\beta_k^l \in \R^{q_l}$ and subject-and-longitudinal outcome specific random effects $b^l \in \R^{r_l} \sim \cN(0, D_{ll})$
  \item<2-> $\text{Cov}[b^l,b^{l'}] = D_{ll'}$
  \only<2>{ and
  \[ D = 
  \begin{bmatrix}
    D_{11} & \cdots & D_{1L}\\
    \vdots &  \ddots & \vdots \\
    D_{1L}^\top & \cdots & D_{LL}
  \end{bmatrix}
  \]
  the global variance-covariance matrix}

\end{itemize}
\end{block}

\begin{block}<3->{Group-specific risk of event}
\begin{itemize}
  \item<3-> $\lambda(t|\cY(t), G = k) = \lambda_0(t) \exp \Big\{\underbrace{\sum_{l=1}^L \sum_{a=1}^\cA \gamma_{k,a}^l \Psi_a^l(t)}_{\gamma_k^\top \Psi(t)} \Big\}$ \\
    \vspace{.1cm}
    with $\cY(t) = \{Y(u), 0 \leq u < t\}$ and for each $l$-th longitudinal outcome, we consider $\cA \in \N_+$ known functionals $\Psi_a^l$ extracted from $\cY^l(t)$ through a given representation mapping
  \item<4-> $\Psi(t) \in \R^{L\cA}$ : high-dim representation features vector, highly flexible and independent of any modeling assumption !
  
\end{itemize}
\end{block}

\end{frame}

\begin{frame}{Generalization of SREMs and JLCMs}

\begin{itemize}
  \item<1-> JLCMs~\citep{vermunt2003latent} : homogeneous latent subgroups sharing the same marker trajectories and the same prognostic
  \item<2-> SREMs~\citep{wulfsohn1997joint} : characteristics of the longitudinal processes (e.g. functions of the random effects) are included as features in the survival model 
  \item<3-> Usual functionals would be
    \begin{table}[htb]
    \centering
    \resizebox{.9\textwidth}{!}{
    \begin{tabular}{ccc}
    \toprule
    Description & $\tilde \Psi_a^l(t, \beta_k^l, b^l)$ & Reference \\
    \midrule
    Linear predictor & $m_k^l(t)$ & \citet{chi2006joint} \\ [.15cm]
    Random effects & $b^l$ & \citet{hatfield2011joint} \\ 
    Time-dependent slope & $\dfrac{\dd}{\dd t} m_k^l(t)$ & \citet{rizopoulos2011bayesian} \\ [.3cm]
    Cumulative effect & $\int_0^t m_k^l(s) \dd s$ & \citet{andrinopoulou2017combined} \\ [.1cm]
     \bottomrule
    \end{tabular}}
    \end{table}
  \item<4-> Depend on $\beta_k$ (then on $k$) which leads to complicated updates, and on $b$ which leads to untracktable integrals that requires approximation methods : computationally intensive and not scalable

\end{itemize}

\end{frame}

\begin{frame}{Graphical model}

\centering
\includegraphics[width=7cm]{figures/graphical_model}

\end{frame}

\subsection{Notations}

\begin{frame}{Notations}

\footnotesize
\begin{itemize}
  \item<1-> $\cD_n = \big\{ (x_1, y_1^1, \ldots, y_1^L, t_1, \delta_1), \ldots, (x_n, y_n^1, \ldots, y_n^L, t_n, \delta_n) \big\}$ with $y_i^l=(y_{i1}^l, \ldots, y_{in_i^l}^l)^\top \in \R^{n_i^l} \text{ and } y_{ij}^l=Y_i^l(t_{ij}^l)$
  \item<2-> $y_i = ({y_i^1}^\top \cdots {y_i^L}^\top)^\top \in \R^{n_i}$ with $n_i = \sum_{l=1}^L n_i^l$
  \item<3-> $b_i = ({b_i^1}^\top \cdots {b_i^L}^\top)^\top \in \R^r$ with $r = \sum_{l=1}^L r_l$
  \item<4-> Design matrices 
  \[ U_i = 
\begin{bmatrix}
  U_{i1} & \cdots & 0\\
  \vdots &  \ddots & \vdots \\
  0 & \cdots & U_{iL}
\end{bmatrix} 
\in \R^{n_i \times q}
\text{ and }
V_i = 
\begin{bmatrix}
  V_{i1} & \cdots & 0\\
  \vdots &  \ddots & \vdots \\
  0 & \cdots & V_{iL}
\end{bmatrix}
\in \R^{n_i \times r}
\]
with $q = \sum_{l=1}^L q_l$ and where for all $l=1, \ldots, L$, one writes
\[
\left\{
    \begin{array}{ll}
        U_{il} &= \big(u_i^l(t_{i1}^l)^\top \cdots u_i^l(t_{in_i^l}^l)^\top\big)^\top \in \R^{n_i^l \times q_l},\\
        V_{il} &= \big(v_i^l(t_{i1}^l)^\top \cdots v_i^l(t_{in_i^l}^l)^\top\big)^\top \in \R^{n_i^l \times r_l}.
    \end{array}
\right.
\]
  \item<5-> $\beta_k = ({\beta_k^1}^\top \cdots {\beta_k^L}^\top)^\top \in \R^q$
  \item<6-> $M_{ik} = U_i\beta_k + V_ib_i \in \R^{n_i}$
\end{itemize}

\end{frame}

\subsection{Likelihood}

\begin{frame}{Likelihood}

\footnotesize
\begin{itemize}
  \item<1-> \scriptsize $\theta = \big(\xi_0^\top \cdots \xi_{K-1}^\top, \beta_0^\top \cdots \beta_{K-1}^\top, \phi^\top, \text{vech}(D), \lambda_0^\top, \gamma_0^\top \cdots \gamma_{K-1}^\top\big) \in \R^\vartheta$ 
  \item<2-> $f(y_i|b_i, G_i=k) = \exp \big\{(y_i \odot \Phi_i)^\top M_{ik} - c_\phi(M_{ik}) + d_\phi(y_i) \big\}$ with $\Phi_i = (\phi_1^{-1} {\textbf{1}_{n_i^1}}^\top \cdots \phi_L^{-1} {\textbf{1}_{n_i^L}}^\top)^\top \in \R^{n_i}$
  \item<3-> Survival part: 
  \begin{align*}f(t_i, \delta_i| G_i = k ; \theta) = &\big[\lambda(t_i|\cY(t_i), G_i = k)\big]^{\delta_i} \\ & \times \exp \Big\{-\int_0^{t_i} \lambda(s|\cY(s), G_i = k) \dd s \Big\}
  \end{align*}
  \item<4-> Then, the log-likelihood (JLCMs type) writes
  \begin{align*}\ell_n(\theta) = n^{-1} \sum_{i=1}^n \log \sum_{k=0}^{K-1} \pi_{\xi_k}(x_i) f_\theta(t_i, \delta_i| G_i = k) f_\theta(y_i | G_i = k) \dd b_i
  \end{align*}
\end{itemize}

\end{frame}

\section{Inference}

\begin{frame}[noframenumbering]
\Large \centering
\textcolor{blue_pres}{III.} Inference
\end{frame}

\subsection{Penalization}

\begin{frame}{Penalization}

\small

\begin{itemize}
  \item<1-> Penalized objective
  \begin{equation*}
  \ell_n^\text{pen}(\theta) = - \ell_n(\theta) + \sum_{k=0}^{K-1} \zeta_{1,k} \norm{\xi_k}_{\text{en}, \eta} + \zeta_{2,k} \norm{\gamma_k}_{\text{sg} l_1, \tilde{\eta}}
  \end{equation*}
  with the elasticnet penalty \[ \norm{z}_{\text{en}, \eta} = (1-\eta)\norm{z}_1 + \dfrac\eta2 \norm{z}_2^2 \] and the sparse group lasso penalty \[ \norm{z}_{\text{sg} l_1, \tilde{\eta}} = (1-\tilde{\eta})\norm{z}_1 + \tilde{\eta} \sum_{l=1}^L\norm{z^l}_2 \]
  \item<2-> Resulting optimization problem \[\hat \theta \in \argmin_{\theta \in \R^\vartheta} \ell_n^\text{pen}(\theta)\]
\end{itemize}

\end{frame}

\subsection{prox-QNEM}

\begin{frame}{prox-QNEM algorithm (1/2)}

\small

\begin{itemize}
  \item<1-> $\ell_n^\text{comp}(\theta) = \ell_n^\text{comp}(\theta ; \cD_n, \textbf{\textit{b}}, \textbf{\textit{G}})$
\end{itemize}

\begin{block}<2->{Monte Carlo E-step}
\begin{itemize}
  \item<2-> $\cQ_n(\theta, \theta^{(w)}) = \E_{\theta^{(w)}}[\ell_n^\text{comp}(\theta) | \cD_n]$
  \item<3-> Requires to compute expectations of the form
  \footnotesize
\[ \E_{\theta^{(w)}}[ g(b_i, G_i) | t_i, \delta_i, y_i] = \sum_{k=0}^{K-1} \pi_{ik}^{\theta^{(w)}} \int_{\R^r} g(b_i, G_i) f_{\theta^{(w)}}(b_i | t_i, \delta_i, y_i) \dd b_i \] 
\small for different functions $g$, where we denote 
\[\pi_{ik}^{\theta^{(w)}} = \P_{\theta^{(w)}}[G_i = k | t_i, \delta_i, y_i] \]
\end{itemize}
\end{block}


\end{frame}

\begin{frame}{prox-QNEM algorithm (2/2)}

\scriptsize

\begin{block}<1->{Proximal Quasi-Newton M-step}
\begin{itemize}
  \item<1-> \tiny $\theta^{(w+1)} \in \argmin_{\theta\in \R^\vartheta} \cQ_n(\theta, \theta^{(w)}) + \sum_{k=0}^{K-1} \zeta_{1,k} \norm{\xi_k}_{\text{en}, \eta} + \zeta_{2,k} \norm{\gamma_k}_{\text{sg} l_1, \tilde{\eta}}$
  \item<2-> $D^{(w+1)} = n^{-1} \sum_{i=1}^n \hat \E_{\theta^{(w)}}[ b_i b_i^\top | t_i, \delta_i, y_i]$
  \item<3-> $P^{(w)}_{n,k}(\xi_k) = -n^{-1} \sum_{i=1}^n \hat \pi_{ik}^{\theta^{(w)}} \log \pi_{\xi_k}(x_i)$
  \item<4-> $\xi_k^{(w+1)} \in \argmin_{\xi_k \in \R^p} P^{(w)}_{n,k}(\xi_k) + \zeta_{1,k} \norm{\xi_k}_{\text{en}, \eta}$ 
  \item<5-> L-BFGS-B to solve the problem
  \tiny
  \begin{equation*}
  \begin{split}
    \text{minimize}& \quad \quad P_{n, k}^{(w)}(\xi_k^+ - \xi_k^-) + \zeta_{1,k} \big((1 - \eta) \sum_{j=1}^p (\xi_{k,j}^+ + \xi_{k,j}^-) + \dfrac \eta 2 \norm{{\xi_k}^+ - {\xi_k}^-}_2^2 \big) \\
    \text{subject to}& \quad \quad \xi_{k,j}^+ \geq 0 \text{ and } \xi_{k,j}^- \geq 0 \text{ for } j = 1, \ldots, p
  \end{split} 
  \end{equation*}
  \item<6-> Closed-form update for $\beta_k^{(w+1)}$
  \item<7-> Proximal gradient (ISTA) for the $\gamma_k^{(w+1)}$ update, based on Lemma 1 that states $\text{prox}_{\text{sg} l_1,\tilde \eta, \zeta} = \text{prox}_{\zeta \tilde \eta \sum_{l=1}^L \norm{\cdot}_{2,l}} \circ \text{prox}_{\zeta (1 - \tilde \eta) \norm{\cdot}_1}$
  \item<8-> $\lambda_0^{(w+1)}(t)= \dfrac{\sum_{i=1}^n \delta_i \ind{\{t=t_i\}}}{\sum_{i=1}^n \sum_{k=0}^{K-1} \hat \pi_{ik}^{\theta^{(w)}} \exp \big\{{\gamma_k^{(w+1)}}^\top \Psi_i(t_i)\big\} \ind{\{t_i \geq t\}}}$
\end{itemize}
\end{block}

\end{frame}

\subsection{Evaluation}

\begin{frame}{Performance evaluation}

\scriptsize
\begin{block}<1->{Real-time prediction paradigm}

\begin{itemize}
 \item<1-> Practitioners need predictive prognostic tools to be used in real-time
 \item<2-> Obtain $\hat\theta$ on training data
 \item<3-> On test data : sample $t^{max}_i \sim T_i \big(1 - \text{Beta}(2, 5)\big)$ the time for subject $i$ when one wants to perform the risk prediction -- the ``present'' time
 \only<3>{
    \centering
    \\
    \vspace{.2cm}
    \includegraphics[width=.3\textwidth]{./figures/beta_law.pdf}
  }
  \item<4-> Truncate $y_i$ up to $t^{max}_i$ to get $\tilde y_i$
\end{itemize}

\end{block}

\begin{block}<5->{Predictive marker}

\begin{itemize}
 \item<5-> $\P_{\hat \theta}[G_i=k | T_i > t^{max}_i, \tilde y_i] = \dfrac{\pi_{\hat \xi_k}(x_i) f_{\hat \theta}(t^{max}_i, \delta_i=0, \tilde y_i | G_i=k)}{\sum_{k=0}^{K-1} \pi_{\hat \xi_k}(x_i) f_{\hat \theta}(t^{max}_i, \delta_i=0, \tilde y_i | G_i=k)}$
 \item<6-> In practice $K=2$
 \item<7-> C-index metric $\cC_\tau =\P[\hat \cR_i > \hat \cR_j | T_i < T_j , T_i < \tau]$
\end{itemize}

\end{block}

\end{frame}

\section{Application}

\subsection{Competitors}

\begin{frame}{Competing models}

\begin{itemize}
 \item<1-> Landmark Cox model : \texttt{survival} \texttt{R} package~\citep{zhang2018time}
 \item<2-> The time-dependent Cox model~\citep{sueyoshi1992semiparametric} : \texttt{survival} \texttt{R} package
 \item<3-> Multivariate joint latent class model : \texttt{R} package \texttt{lcmm}~\citep{2017_lcmm}
 \item<4-> Multivariate shared random effect model : \texttt{JMbayes} package~\citep{2017_JMbayes}
\end{itemize}

\end{frame}

\subsection{Simulations}

\begin{frame}{High-dimensional simulation study (1/2)}

\begin{itemize}
 \small
 \item<1-> High-risk subjects proportion $\pi_1 \in [0,1]$
 \item<2-> $\cH = \big\{\lfloor \pi_1 n \rfloor \text{ random samples without replacement} \big\}$
 \item<3-> $[x_{ij}] \in \R^{n \times p} \sim \cN \big(0, \bSigma_1(\rho_1)\big)$
 \item<4-> $x_{ij} \leftarrow x_{ij} + (-1)^{\ind{\{i \notin \cH\}}} gap\ \text{for } j = 1, \dots, s$
 \item<5-> $\xi = (\underbrace{\varsigma_1,\ldots,\varsigma_1}_s,0,\ldots,0) \in \R^p$
 \item<6-> $G_i \sim \cB\big(\pi_\xi(x_i)\big)$
 \item<7-> $\cS_k=\Big\{ k \big\lfloor \dfrac{L r_s}{K} \big\rfloor + 1, \ldots,  (k + 1) \big\lfloor \dfrac{L r_s}{K} \big\rfloor \Big\}$
 \item<8-> $\beta_k^l \sim \ind{\{l \in \cS_k \}} \cN\Big( 
\mu_k, 
\begin{bmatrix}
  \rho_3 & 0\\
  0 & \rho_3
\end{bmatrix} 
\Big)$
 \item<9-> $Y_i^l(t) = \epsilon_i^l(t) + \ind{\{l \in \cS_k \}}\sum_{k=0}^{K-1} \ind{\{G_i=k\}} \big( (1, t)^\top \beta_k^l + (1, t)^\top b_i^l \big)$
 \item<10-> $\big(\tilde \Psi_{k,a}(t, \beta_k, b_i^l)\big)_{a \in \{1,2\}} = \big(\beta_{k,1}^l + \beta_{k,2}^l t + b_{i,1}^l + b_{i,2}^l t\ , {b_i^l}^\top \big)$
 \item<11-> $\tilde \gamma_{k,a}^l = \varsigma_2 \ind{\{l \in \cS_k,\ a\in \{1, 2\} \}}$
 \item<12-> $\lambda_i(t|G_i = k) = \lambda_0(t) \exp \big\{ \iota_{i,k,1} + \iota_{i,k,2} t \big\}$
 \item<13-> Gompertz baseline $\lambda_0(t) = \kappa_1 \kappa_2 \exp(\kappa_2t)$
 \item<14-> $T_i^\star | G_i=k \sim \dfrac{1}{\iota_{i,k,2} + \kappa_2} \log \Big(1 - \dfrac{(\iota_{i,k,2} + \kappa_2) \log U_i}{\kappa_1 \kappa_2 \exp\iota_{i,k,1}} \Big)$
\end{itemize}

\end{frame}

\begin{frame}{High-dimensional simulation study (2/2)}

\centering
\only<1>{\includegraphics[width=1\textwidth]{./figures/simu_param}}
\only<2>{
  \includegraphics[width=.9\textwidth]{./figures/simu_time}
  \includegraphics[width=.7\textwidth]{./figures/simu_surv}
  }
\only<3>{\includegraphics[width=1\textwidth]{./figures/simu_long}}

\end{frame}


\subsection{Results}

\begin{frame}{First results (1/2)}

\centering
\includegraphics[width=1\textwidth]{./figures/obj_vcg}
\uncover<2->{\includegraphics[width=.5\textwidth]{./figures/xi_estimation}}
\uncover<3->{\includegraphics[width=.45\textwidth]{./figures/baseline_estimation}}
\uncover<4->{\includegraphics[width=.8\textwidth]{./figures/beta_estimation}}

\end{frame}

\begin{frame}{First results (2/2)}

\centering
\includegraphics[width=1\textwidth]{./figures/gamma_estimation}
\uncover<2->{\includegraphics[width=1\textwidth]{./figures/predictions}}

\end{frame}


\section{Conclusion}

\begin{frame}[noframenumbering]
\Large \centering
\textcolor{blue_pres}{V.} Conclusion
\end{frame}

\begin{frame}{Conclusion}

\begin{itemize}
  \item<1-> Prognostic method called lights introduced to deal with the problem of joint modeling of longitudinal data and censored durations, where a large number of longitudinal features are available
  \item<2-> Penalization of the likelihood in order to perform feature selection and to prevent overfitting
  \item<3-> New efficient estimation algorithm (prox-QNEM) has been derived
  \item<4-> Automatically determines significant prognostic longitudinal features
\end{itemize}

\begin{block}<5->{\texttt{Python} 3 package}
\begin{itemize}
  \item<5-> Available at \small{\url{https://github.com/Califrais/lights}}
  \item<6-> Applications of the model available soon
\end{itemize}
\end{block}

\end{frame}

\begin{frame}[noframenumbering]
\Large \centering
\textcolor{blue_pres}{} Thank you!
\end{frame}

\section{References}
\frame{
  \tiny
  \frametitle{References}
  \bibliographystyle{plainnat}
  \bibliography{refs}
}

\end{document}
              